\documentclass[11pt]{article}

\usepackage[utf8]{inputenc}
\usepackage[ngerman]{babel}
\usepackage{csquotes}

\usepackage{fullpage}
\usepackage{setspace}
\usepackage{parskip}
\usepackage{titlesec}
\usepackage{mathptmx}

\usepackage{graphicx}

\PassOptionsToPackage{
style=apa,
doi=false,
isbn=false,
eprint=false
}{biblatex}
\usepackage[backend=biber]{biblatex}
\addbibresource{library.bib}

\PassOptionsToPackage{hyphens}{url}

\makeatletter

\renewenvironment{abstract}
{{\bfseries\noindent{\abstractname}\par\nobreak}\footnotesize}
{\bigskip}

\renewenvironment{quote}
  {\begin{tabular}{|p{13cm}}}
  {\end{tabular}}

\titlespacing{\section}{0pt}{*3}{*1}
\titlespacing{\subsection}{0pt}{*2}{*0.5}
\titlespacing{\subsubsection}{0pt}{*1.5}{0pt}

\usepackage{authblk}

%\usepackage{longtable}
%\usepackage{tabulary}
%\usepackage{booktabs,array,multirow}
\usepackage[colorlinks = false]{hyperref}

\begin{document}

\begin{titlepage}
   \begin{center}
       \vspace*{1cm}
       
       \Huge	
       Zusammenhang zwischen sportlichem Erfolg und Darstellung bei Wikipedia im Vergleich zwischen Sportlern und Sportlerinnen
 
       \vspace{2.0cm}
       
       \includegraphics[width=0.4\textwidth]{logo.jpg}
 
       \vspace{1.5cm}
       \LARGE
       Josefine Busch, Lisa Hillebrand, Flip Jansen, Daniela Tumbrägel
 
       \vfill
 
       Grundlagen sozialer Netze \\
       Prof. Dr. Gefei Zhang\\
       Wintersemester 2018/2019\\
 
       \vspace{0.8cm}
      
       Hochschule für Technik und Wirtschaft\\
       Berlin\\
 
   \end{center}
\end{titlepage}



\author{Lisa Hillebrand}%
\author{Josefine Busch}%
\author{Daniela Tumbrägel}%
\author{Flip Jansen}%
\affil{HTW Berlin}%

%\date{\today}



%\begingroup
%\let\center\flushleft
%\let\endcenter\endflushleft
%\maketitle
%\endgroup

\pagebreak

\begin{abstract}
\textbf{}---%
\end{abstract}%

\section {Einleitung, Problemformulierung // Tumbrägel, Daniela (558529)}

Die vorliegende Studie ist eine explanative Analyse des Zusammenhangs zwischen sportlichem Erfolg von Sportler*innen und deren Repräsentation auf Wikipedia, gemessen an der Länge der Artikel und der Anzahl der Editierungen. Wie man am Beispiel von Caster Semenya feststellen muss, sind die Olympischen Spiele noch sehr von binären Geschlechterkonstrukten geprägt und unterscheiden die Teilnehmer*innen in ihren Wettbewerben dementsprechend \parencite{IOCRules}. Daher ist die vorliegende Studie zwangsläufig dieser limitierten Sichtweise angepasst und unterscheidet nur zwischen den Kategorien Frauen und Männern.

Die Motivation für die Studie ergab sich aus dem Fall von Donna Strickland, die im Jahr 2018 - zusammen mit Gérard Mourou und Arthur Ashkin - einen Nobelpreis im Bereich der Laserphysik erhalten hat \parencite{nobelprize}. Sie ist die dritte Frau, der je ein Physik-Nobelpreis verliehen wurde, und erst nach der Anerkennung wurde ein Wikipedia-Artikel über sie veröffentlicht \parencite{Bazely2018}. Ihre männlichen Kollegen sind in der Online-Enzyklopädie schon seit mehreren Jahren vertreten. Einige Monate zuvor wurde ein Eintrag über Strickland abgelehnt, da sie von einem der Wikipedia-Editoren als nicht wichtig genug erachtet wurde \parencite{stricklandWiki}. Neben zahlreichen anderen Beispielen zeigt dies die geschlechtsspezifische Diskriminierung auf Wikipedia und insbesondere die Ausgrenzung von Frauen in der Wissenschaft auf \parencite{LeylandCecco}.

Geplant war, durch unsere Untersuchung zu überprüfen, ob Frauen in der Wissenschaft auf Wikipedia anders repräsentiert werden als Männer. Bei der Suche nach effizienten Datenquellen und Datensätzen ergab sich allerdings die Problematik, dass Frauen in der Wissenschaft insgesamt weniger vertreten sind und es somit sehr schwierig ist, einen sinnvollen Vergleich zwischen Frauen und Männern herzustellen. Deshalb ist die Entscheidung für einen anderen Themenbereich gefallen, in dem Frauen und Männer vergleichbaren Maßen vertreten sind: Sport. Hier ist es möglich ein klares Ranking von individuellen Leistungen aufzustellen und Teilnehmer*innen auf dieser Grundlage zu vergleichen.

"Research shows just 16\% of Wikipedia editors are female and only 17\% of entries dedicated to notable people are for women" \parencite{PoppyNoor}.

Die Online-Enzyklopädie Wikipedia wird immer wieder für diverse Vorurteile und Tendenzen kritisiert. Darunter fallen Kategorien wie Gender, Kultur und Sprache sowie Geschichte und Politik \parencite{PoppyNoor}.. Diese Studie konzentriert sich dabei auf die Kategorie Gender und untersucht diese im Fokus der Olympischen Sommerspiele 2016 in Rio de Janeiro.

Im Hinblick auf die größtenteils gleichmäßige Anzahl von Frauen und Männern wird das Verhältnis von sportlichem Erfolg und der entsprechenden Darstellung auf Wikipedia untersucht. Motiviert durch die oben genannten Ereignisse, bildet sich unsere These, dass Frauen bei gleichzusetzender Leistung im Vergleich zu Männern auf Wikipedia unzureichend repräsentiert werden. Im Detail wird dabei die mittlere Länge und die mittlere Anzahl der redaktionellen Änderungen der Wikipedia-Artikel der Sportler*innen differenziert.

\section*{Theorie}
\label{intro}
\subsubsection*{Wikipedia}
Wikipedia wurde im Jahr 2001 gegründet \parencite{wikipediaTimeline}. Mittlerweile gibt es sprachenübergreifend über 46 Millionen Artikel \parencite{wikipedia_Size}. Von rund 35 Millionen registrierten Nutzer:innen sind jedoch nur etwa 124.000 monatlich aktiv \parencite{wikipedians}. 
Wikipedia ist zum Zeitpunkt des Verfassens dieses Texts auf dem fünften Platz der weltweit populärsten Websites \parencite{Alexa2019} und hat damit eine enorme Reichweite.
Jede Person kann zu Wikipedia beitragen und auch ohne Registrierung Artikel bearbeiten \parencite{wikipediaTutorial}. Das Erstellen neuer Artikel ist nur nach vorheriger Registrierung möglich. Jeder Artikel verfügt über eine Versionsgeschichte, in der vorherige Artikelversionen sowie die Anmerkungen der Autor:innen und Administrator:innen gespeichert werden.

\subsubsection*{Gender Bias}

Ein Bias ist eine Verzerrung bzw. ein systematischer Fehler \parencite{Wirtz}. Als Gender Bias bezeichnet man eine Bevorzugung oder Parteilichkeit gegenüber einer oder mehreren Personen ausschließlich wegen des Geschlechts. Es sollen nun einige Erkenntnisse hinsichtlich des Gender Bias im Sport zusammengefasst und und daraus die Hypothesen abgeleitet werden.

\subsubsection{Gender Bias und Wikipedia}
In mehreren Studien konnte gezeigt werden, dass ein Gender Bias sowohl in der Zusammensetzung der Autorenschaft bei Wikipedia als auch hinsichtlich der inhaltlichen Gestaltung der Artikel besteht. 
Unter den Autoren sind bestimmte Personengruppen deutlich unterrepräsentiert. In Nutzerbefragungen wurde erfasst, dass zwischen 84\% und 90\% aller Wikipidia-Autor:innen männlich sind \parencite{wikimediaReport,GraellsGarrido2015}. Hierzu muss jedoch angemerkt werden, dass die von der Wikimedia Foundation selbst durchgeführten Erhebungen Opt-in-Befragungen sind, die zu Verzerrungen führen können. So ist vorstellbar, dass weniger Frauen als Männer an der Befragung teilnehmen, sie aber eigentlich unter den Autoren deutlich häufiger vertreten sind. \textcite{Hill2013} überprüfen genau diese Kritikpunkte und berechnen einen korrigierten Wert für den Anteil Autorinnen, der jedoch auch nach Korrektur bei lediglich 16.1\% liegt. Es lässt sich also zusammenfassen, dass nach jetzigem Stand der Kenntnis deutlich weniger Frauen als Männer zu Wikipedia beitragen.

Auf inhaltlicher Ebene untersuchten verschiedene Studien das Vorliegen eines Gender Bias. \textcite{Wagner2015} analysierten Wikipedia-Artikel über bekannte Personen in vier Parametern: Abdeckungsbias (coverage bias), Strukturbias (structural bias), Lexikalischer Bias (lexical bias) und Sichtbarkeitsbias (visibility bias). Der Abdeckungsbias beschreibt, über wie viele Personen Wikipedia-Artikel bestehen. So könnte eine Annahme sein, dass über bekannte Frauen weniger Artikel verfasst werden als über bekannte Männer. Mit dem Strukturellen Bias werden Unterschiede beispielsweise in der Verlinkungsstruktur beschrieben. Ein Bias würde bestehen, wenn Frauenartikel mehr Verlinkungen auf Artikel über Männer enthalten als umgekehrt. Ein Lexikalischer Bias umfasst Unterschiede im Vokabular, mit dem Männer und Frauen beschrieben werden. Ein Beispiel für einen Bias wäre, dass in Artikeln über Frauen mehr Vokabular mit Beziehungs- oder Familienbezug verwendet wird. Der Sichtbarkeitsbias gibt an, ob es Unterschiede zwischen den Geschlechtern in der Häufigkeit gibt, mit der Artikel auf der Startseite von Wikipedia platziert werden.
In ihrer Studie konnten \textcite{Wagner2015} Evidenz für den Strukturellen und den Lexikalischen Bias finden. Artikel über Frauen verlinken häufiger auf Artikel von Männern als umgekehrt. Auf lexikalischer Ebene konnten die Autoren zeigen, dass in Artikeln über Frauen mehr Vokabular mit Bezug zu Familie und romantischen Beziehungen verwendet wird als in Artikeln über Männern. Ein Abdeckungsbias konnte nicht gezeigt werden; bekannte Frauen und Männer, die für die Studie ausgewählt wurden, waren quantitativ gleich gut vertreten.

In einer Studie von  \textcite{GraellsGarrido2015} fand sich hingegen, dass der Gender Bias auf Wikipedia mit Struktur und Inhalt von Artikeln zusammenhängt. In der Artikellänge besteht ein signifikanter Unterschied, wobei Artikel über Frauen im Mittel kürzer sind als Artikel über Männer. Die Effektstärke fiel lediglich gering aus.

Die vorliegende Untersuchung fokussiert auf den Abdeckungsbias und untersucht die Artikellänge sowie die Anzahl der Editierungen.

\subsubsection{Gender Bias im Sport}

In einigen Sportarten erhalten Sportler mehr Aufmerksamkeit als Sportlerinnen, was sich in den Umsätzen und schlussendlich auch in den Gehältern niederschlägt. So liegt das Durchschnittsgehalt in der Fußball-Bundesliga der Männer bei rund 1,5 Millionen Euro \parencite{Harris2015}, während das Durchschnittsgehalt der Frauen bei 43.000 Euro pro Jahr liegt \parencite{soccerIncomeWomen}. Zahlreiche Studien haben sich mit der Medienberichterstattung über Sportler und Sportlerinnen befasst. Zum einen gibt es Sportarten, in denen weitaus mehr Zeit der TV-Berichterstattung für die männlichen Sportler aufgewendet wird - Fußball oder Basketball sind hier nur zwei Beispiele. In der NBC-Berichterstattung über die Olympischen Spiele 2008 wurde 54,2\% der Übertragungszeit über Sportler berichtet und 44,8\% der Zeit über Sportlerinnen \parencite{Billings2008}. \textcite{Trolan2013} fassten Erkenntnisse anderer Studien zusammen und kommen zu dem Schluss, dass Frauen in der Medienberichterstattung sowohl quantitativ unterrepräsentiert sind als auch qualitativ anders berichtet wird. Die Leistungen von Frauen werden trivialisiert, indem ein stärkerer Fokus auf das Äußere von Sportlerinnen gelegt wird als auf ihre Leistungen \parencite{Harris2005,Vincent2004}. \textcite{Jones1999} untersuchten die Berichterstattung über die Olympischen Spiele 1998 und fanden, dass über Sportlerinnen mehr berichtet wird, wenn sie einer Sportart nachgehen, die als weniger maskulin beurteilt wird. \textcite{Billings2008} untersuchten, wie Sportkommentator*innen Erfolg und Misserfolg bei Sportler*innen attribuieren. Im Falle von Erfolg führten sie diesen bei Sportlern signifikant häufiger auf deren überlegene körperliche Stärke zurück.
Zusammenfassend lässt sich also sagen, dass es quantitative und qualitative Unterschiede in der Berichterstattung über Sportler und Sportlerinnen gibt. Daraus abgeleitet ergibt sich die Annahme, dass sich dies auch in der Aufmerksamkeit niederschlägt, die Sportler*innen bei Wikipedia erfahren.

\subsubsection*{Hypothesen}

Es ergeben sich folgende Hypothesen:

\textbf{Hypothese 1}. Wikipedia-Artikel über Sportler sind signifikant länger als Wikipedia-Artikel über Sportlerinnen.

\textbf{Hypothese 2}. Wikipedia-Artikel über Sportler wurden signifikant häufiger editiert als Wikipedia-Artikel über Sportlerinnen.

Diese Untersuchung überprüft die Hypothesen anhand von acht ausgewählten Wettkämpfen bei den Olympischen Spielen in Rio 2016.

\section*{Methode //Josefine, Busch 560106}
\label{igw}

\subsection*{Erhebungsart}
Zur Beschaffung der Daten stehen uns zwei Möglichkeiten zur Verfügung. Auf der offiziellen Webseite der Olympischen Kommission sind die Namen der Teilnehmenden der vergangenen Spiele veröffentlicht.\cite{olympicResults} Der Vorteil dieser Daten ist, dass sie bereits in weibliche und männliche Teilnehmer unterteilt sind. Alternativ sind auch direkt auf Wikipedia Listen der Teilnehmenden zu finden. \cite{wikiOlympicComp} Die Wikipedia Listen beinhalten direkte Verlinkungen zu den jeweiligen Seiten der Sportler*innen, allerdings ist hier nicht mehr ohne Weiteres auszulesen, welchem Gender die Person angehört. Um diese Information zu bekommen müsste der Fließtext auf die verwendeten Pronomen untersucht werden. Auch andere Informationen sind bei Wikipedia nicht mehr so leicht auslesbar wie aus der Liste von Olympic.org. Besonders sind die Sportler*innen nach dem Alphabet geordnet nicht nach dem erreichten Rang, dieser müsste also ebenfalls aus dem Fließtext ausgelesen werden.

Die Liste der Olympischen Spiele 2016 von Olympic.org enthält strukturierte Daten in Form der folgenden Spalten:

\begin{itemize}
\item Year 
\item Sport
\item Discipline
\item Event
\item EventGender
\item Phase
\item PhaseXX
\item Unit
\item UnitXX
\item Names
\item Gender
\item NOC
\item Teammembers
\item Rank
\item Results
\item Medal
\item Adversary
\item Adversary NOC
\end{itemize}

Die für uns interessanten Daten hieraus sind die Spalten Sport, Discipline, Event, Names und Gender.
Die Länge der Artikel, gemessen an der Anzahl der Wörter, und die Anzahl der Editierungen der einzelnen Artikel müssen erst berechnet beziehungsweise von Wikipedia direkt abgefragt werden. Diese Daten liegen in unstrukturierter Form vor.

\subsection*{Stichproben}
Da nicht in allen Sportarten Männer und Frauen in vergleichbaren Disziplinen antreten und in manche Disziplinen Teams gegeneinander antreten, wählen wir acht Disziplinen aus unterschiedlichen Sportarten aus, welche unsere Anforderungen erfüllen. Die Anforderungen sind, dass die Disziplinen für Männer und Frauen gleich sind bzw. gleich bezeichnet sind und dass sie keine Teamdisziplinen sind.
Die ausgewählten Sportarten sind:
\begin{itemize}
\item Turmspringen (10m)
\item Bogenschießen
\item Fechten (épée)
\item Moderner Fünfkampf
\item Stabhochsprung
\item Schwimmen (100m Freestyle)
\item Radfahren
\item Athletik (100m)
\end{itemize}

Um die Artikel von Wikipedia zu bekommen und von diesen die Länge in Wörtern verwenden wir ein Modul von GitHub Benutzer Jonathan Goldsmith \cite{goldsmith}. Mit diesem Modul und den Titeln der Wikipedia Seiten können die Artikel abgerufen werden. Danach musst nur noch die Wörteranzahl ermittelt werden. Mit Stichproben soll geprüft werden, dass die korrekten Artikel zurück gegeben und gezählt werden.

Für die Zwischenspeicherung der Daten im Jupyter Notebook verwenden wir Pandas dataFrames. Außerhalb des Programms sind die Daten in CSV-Dateien abgelegt. Für die grafische Darstellung im Notebook verwenden wir außerdem Matplotlib\parencite{matplotlib} und Seaborn\parencite{seaborn}.

\subsection*{Auswertungsmethoden}

Mit Histogrammen zu den unterschiedlichen Disziplinen wollen wir die Verteilung der Anzahl der Wörter für männliche und weibliche Sportler vergleichen und gegebenenfalls einen T-Test durchführen. Zur weiteren Untersuchung der Extremwerte, des Mittelwerts und des Median sollen Boxplots und Balkendiagramme zu den verschiedenen Disziplinen erstellt werden.

\subsection*{Data Privacy}

---------------Aus Prof. Zhangs Skript ------------------

Methodologie / Untersuchung
    Datenbeschaffung
    Datenpräparation
    Exploration/Modell Planung (Ziel: Hypothese)
    Model-Erstellung (Ziel: Test Hypothese, ggfs. Anpassung des Models)
    
    Wie gehen Sie vor und warum? Welche Werkzeuge nutzen Sie?
§ Datenbeschaffung (Qualität und Quantität der Daten mit
Darstellung weiterer Quellen / Erhebungsmethoden?)
» Erhebungsart
» Stichprobe
» Auswertungsmethoden
§ Ethische Kriterien berücksichtigen (vgl. Data Privacy)
» Missbrauch personenbezogener Daten (Anonymisierung,
Pseudonymisierung)

\section {Durchführung // Jansen, Flip 558059}

\subsection {Datenbeschaffung und Datenpräparation}
Ausgehend von der englischsprachigen Wikipedia-Übersichtsseite der Teilnehmer*innen der Olympischen Sommerspiele 2016 \parencite{wikiOlympicComp} haben wir rekursiv alle Sportarten gesammelt. Nachfolgend haben wir über die Sportarten die URLs und damit die Wikipedia-Titel der einzelnen Sportler*innen einer Disziplin gesammelt. 

Für das Verarbeiten der HTML-Seiten haben wir die Python-Bibliothek \textit{lxml} \parencite{lxml} verwendet.

Um mögliche Fehler beim Sammeln der Daten von \textit{olympic.org} zu vermeiden, haben wir das Olympic Studies Centre per Mail kontaktiert. Diese haben uns daraufhin eine Excel-Tabelle mit den ausführlichen Ergebnissen der Olympischen Sommerspiele 2016 zur Verfügung gestellt.

Die Daten haben wir anschließend nach den von uns gewählten Stichproben gefiltert, sodass daraus 8 Disziplinen mit den jeweils am besten Frauen und Männern resultierten. Da die Anzahl der Frauen und Männer innerhalb einer Disziplin teils nicht identisch war, entfernten wir die überzähligen Sportler*innen einer Disziplin, um gleiche Gruppengrößen innerhalb jeder Disziplin zu erzielen.

Um die Wikipedia-Titel mit den von uns benötigten Daten der Ergebnistabelle zu verbinden, mussten wir nun die unterschiedlichen Namensformate anpassen und Sonderzeichen umwandeln. So setzt zum Beispiel Wikipedia bei Namensgleichheit von in Wikipedia erfassten Personen eine Kategoriebezeichnung ans Ende des Namens, z.B. "(diver)".

Trotzdem konnten wir einige URLs nicht automatisiert zuordnen, da es zum einen in der Ergebnistabelle Rechtschreibfehler in den Namen gab, zum anderen bei Namen aus Sprachgebieten mit nicht-lateinischer Schrift unterschiedliche Umschriften möglich sind. Auch wurde die Reihenfolge von Namensbestandteilen insbesondere aus dem asiatischen Raum nicht einheitlich gehandhabt. Teilweise ließ sich dieses Problem dadurch lösen, die Namensbestandteile permutierend zu vergleichen. 

Ein Versuch, möglichen Rechtschreibfehlern bzw. Problemen mit Sonderzeichen zu begegnen, haben wir mithilfe des Levenshtein-Algorithmus unternommen \parencite{Sulzberger2019}. Dieser gibt die Anzahl von Einfüge-, Lösch- und Ersetz-Operationen zurück, die notwendig sind, um eine Zeichenkette in eine zweite umzuwandeln. Trotz niedrig angesetzten Schwellenwerten gab es aber zu viele falsche Zuordnungen, sodass wir darauf verzichtet haben. Für eine umfangreichere Datenauswertung lohnt es sich vermutlich, weiter in diese Richtung zu forschen. 

Schlussendlich mussten wir einige Wikipedia-URLs der Sportler*innen manuell den entsprechenden Namen zuordnen. 

Mit den bereinigten Daten konnten wir nun mit Hilfe der Wikipedia-Bibliothek für Python \cite{goldsmith} den zum Titel gehörenden Artikel auslesen. Nach Bereinigung des Artikels von Steuerungszeichen haben wir die Wörter gezählt und unserem \textit{pandas} Dataframe hinzugefügt. 

Die Anzahl der Editierungen konnten wir auf der "Page information"-Seite des jeweiligen Hauptartikels auslesen, die diesem Format folgt:
\url{https://en.wikipedia.org/w/index.php?title=PYTHON&action=info}

Mit manuellen Stichproben haben wir die Richtigkeit unserer Daten überprüft.

\section {Ergebnisse}

\section {Diskussion}

Die Ergebnisse zeigen, dass lediglich in der Disziplin Radsport (Straße) die Artikel über Sportler signifikant länger sind als die über Sportlerinnen. Bei der Anzahl der Editierungen gibt es in den Disziplinen Radsport (Straße) und 100m-Sprint mehr Editierungen der Artikel über Sportler. Die Nullhypothesen besagen, dass es keinen Unterschied in der Artikellänge und der Anzahl der Editierungen in den Artikeln über Sportler und Sportlerinnen gibt. In Anbetracht dessen, dass nur in einer bzw. in zwei der acht Kategorien ein signifikantes Ergebnis gefunden wurde, kann keine der Nullhypothesen verworfen werden.
Das gefundene Ergebnis deckt sich mit den Erkenntnissen der Studie von \textcite{GraellsGarrido2015}, in denen zwar signifikant längere Artikel über Männer gefunden wurden bei jedoch nur geringer Effektstärke. Es kann daher der Schluss gezogen werden, dass Männer und Frauen im Sport quantitativ gleich beschrieben werden.

\subsection {Limitationen}

Die Untersuchung beschränkt sich auf den Abdeckungsbias. Es wäre interessant, die Verlinkungsstruktur der Artikel sowie inhaltliche Aspekte zu untersuchen. Wie in zuvor genannten Studien beschrieben gibt es qualitative Unterschiede in der Art, wie Sportreporter*innen den Erfolg oder Misserfolg von Sportler*innen kommentieren. Genauso bestehen Unterschiede in  geschriebenen Artikeln über Sportereignisse. Für eine Inhaltsanalyse wäre es nötig, sich auf die erfolgreichsten Sportler*innen bei Olympia zu beschränken, da die Artikellängen über weniger erfolgreiche Sportler*innen keine Inhaltsanalyse hergeben.
Zudem war es uns leider aufgrund der Datenlage nicht möglich zu untersuchen, inwiefern Wissenschaftler und Wissenschaftlerinnen bei gleicher Leistung auf Wikipedia sowohl quantitativ als auch qualitativ unterschiedlich repräsentiert werden. Es wäre dazu notwendig, ein Ranking von Wissenschaftler*innen aufzustellen, um nur Personen zu vergleichen, die eine ähnliche Reputation genießen. Ein Ansatz wäre, dies über den sogenannten H-Index zu tun, der ein Indikator für die wissenschaftliche Leistung ist\parencite{hIndex}.

\printbibliography
\end{document}

